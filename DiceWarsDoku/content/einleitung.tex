
\chapter{Einleitung}  
F�r die Durchf�hrung des Projekts haben wir uns entschlossen ein Spiel zu
programmieren. Bei dem Spiel handelt es sich um das Spiel DiceWars. Dieses Spiel
wurde also nicht neu erfunden sondern einem Bestehenden nachempfunden.\\
Das Spiel hat sehr viel �hnlichkeit mit dem Brettspiel Risiko. Dort m�ssen
taktisch Einheiten auf Feldern verteilt werden um so seine Feinde zu besiegen.
So auch in unserem DiceWars. Den eigentlichen Spielablauf beschreiben wir in
Kapitel 2 "Spielregeln". \\
Wichtiger Punkt bei der Implementierung war es verschiedene Technologien
einzubinden. So greifen wir mit Hilfe der Random.org API auf einen
Zufallsgenerator zu, der anhand des Weltraumrauschens Zufallszahlen berechnet.\\
Als weitere Technologie verwenden wir eine lokale Datenbank. In dieser Datenbank
k�nnen die Spieler sich verewigen. Hier werden verschiedene Parameter
abgespeichert und diese werden am Ende des Spiel grafisch angezeigt.\\
Damit alle Funktionen sinnvoll eingesetzt werden k�nnen ist die Verwendung
verschiedener Design-Pattern Pflicht. Hier haben wir versucht an m�glichst
sinnvollen Stellen Design-Pattern f�r bekannte Probleme einzusetzen. Dazu jedoch
mehr in dem Kapitel "Design-Pattern". 

\chapter{Verwendete Technologien}  

Random.org API\\ \\

Um beim W�rfeln richtige und keine Pseudo-Zufallszahlen zu bekommen, 
greifen wir auf die Random.org API zu. Diese liefert Zufallszahlen 
in Abh�ngigkeit vom Weltraumrauschen und wird deshalb von Lotterien, 
Casinos und Online-Spielen benutzt, da es sich um wahre Zufallszahlen handelt 
und nicht um Pseudo-Zufallszahlen, die ein Computer generiert.\\
Die Implementierung findet sich in der Klasse RandomGenerator. 
Diese Klasse enth�lt die �ffentliche Methode "public int rollTheDices(int dices)". 
Diese Methode ruft random.org auf mit passenden �bergabeparametern (kleinst m�gliche Zahl, 
gr��t m�gliche Zahl, Anzahl Zahlen) auf. 
Das Ergebnis (bzw. der HTML Code der Ergebnis-Seite) wird mit der Methode "DownloadData" heruntergeladen. 
Dieser muss dann erst noch geparst werden, also die generierten Zahlen aus dem HTML-Code herausgefilert 
und in einem Integer-Array gespeichert werden. Da es bei diesem Spiel nur um die Summe aller Augenw�rfel geht, 
werden alle Zahlen des Integer-Arrays anschlie�end addiert und die Summe als R�ckgabewert zur�ckgegeben.\\
Sollte dabei ein Fehler auftreten (z.B. weil der Rechner gerade keine Internetverbindung hat),
wird die rekursive Methode "rollTheDiceOffline" aufgerufen, die Pseudo-Zufallszahlen generiert 
und die Summe davon zur�ckgibt. Somit ist man nicht auf eine Internetverbindung beim Spielen angewiesen. 